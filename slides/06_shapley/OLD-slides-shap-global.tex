\documentclass[11pt,compress,t,notes=noshow, aspectratio=169, xcolor=table]{beamer}

\usepackage{../../style/lmu-lecture}
% Defines macros and environments
% This file is included in slides and exercises

% Rarely used fontstyle for R packages, used only in 
% - forests/slides-forests-benchmark.tex
% - exercises/single-exercises/methods_l_1.Rnw
% - slides/cart/attic/slides_extra_trees.Rnw
\newcommand{\pkg}[1]{{\fontseries{b}\selectfont #1}}

% Spacing helpers, used often (mostly in exercises for \dlz)
\newcommand{\lz}{\vspace{0.5cm}} % vertical space (used often in slides)
\newcommand{\dlz}{\vspace{1cm}}  % double vertical space (used often in exercises, never in slides)
\newcommand{\oneliner}[1] % Oneliner for important statements, used e.g. in iml, algods
{\begin{block}{}\begin{center}\begin{Large}#1\end{Large}\end{center}\end{block}}

% Don't know if this is used or needed, remove?
% textcolor that works in mathmode
% https://tex.stackexchange.com/a/261480
% Used e.g. in forests/slides-forests-bagging.tex
% [...] \textcolor{blue}{\tfrac{1}{M}\sum^M_{m} [...]
% \makeatletter
% \renewcommand*{\@textcolor}[3]{%
%   \protect\leavevmode
%   \begingroup
%     \color#1{#2}#3%
%   \endgroup
% }
% \makeatother


\title{Interpretable Machine Learning}
% \author{LMU}
%\institute{\href{https://compstat-lmu.github.io/lecture_iml/}{compstat-lmu.github.io/lecture\_iml}}
\date{}

\begin{document}

% \newcommand{\titlefigure}{figure/sample-dgp-2d.pdf}
\newcommand{\learninggoals}{
\item Understand structure of tabular data in ML
\item Understand difference between target and features
\item Understand difference between labeled and unlabeled data
\item Know concept of data-generating process}

\lecturechapter{Shapley Values Aggregation}
\lecture{Interpretable Machine Learning}

\begin{vbframe}{Shapley Values Aggregations}

  \begin{itemize}
    \item Shapley Values are local explanations.
    \item When computed for many observations, can be aggregated to global explanations.
      \begin{itemize}
        \item Feature Importance
        \item Summary plots
        \item dependence plots
        \item Interactions plots
        \item clustered Shapley values
      \end{itemize}
  \end{itemize}
\end{vbframe}

\begin{vbframe}{Feature Importance}
\end{vbframe}

\begin{vbframe}{Summary Plot}
\end{vbframe}

\begin{vbframe}{Dependence Plot}
\end{vbframe}

\begin{vbframe}{Interaction Plots}
\end{vbframe}

\begin{vbframe}{Clustered Shapley values}
\end{vbframe}





\endlecture
\end{document}
