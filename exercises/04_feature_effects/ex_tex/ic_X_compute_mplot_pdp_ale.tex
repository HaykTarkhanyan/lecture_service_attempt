\aufgabe{Understanding M, PD, and ALE Plots}{

You are given a dataset with two features \(x_1, x_2 \in \mathbb{R}\) and a target variable \(y\), where:
\[
y = -x_1 + 2 \cdot x_2 + \epsilon, \quad \epsilon \sim \mathcal{N}(0, \sigma_\epsilon^2).
\]
The features \(x_1\) and \(x_2\) are jointly normally distributed with
$
x_1 \sim \mathcal{N}(0, \sigma^2), x_2 \sim \mathcal{N}(0, \sigma^2)$ (i.e., same variance $\sigma^2$) and $\text{Cov}(x_1, x_2) = \rho \cdot \sigma^2
$ with Pearson correlation coefficient $\rho \approx 1$ (i.e., both features are highly correlated). %and we have $x_1 \approx x_2$
The model prediction function is:
$
\hat{f}(x_1, x_2) = -x_1 + 2 \cdot x_2.
$

Your task is to calculate and analyze the following plots for feature \(x_1\):

\begin{enumerate}

\item \textbf{M Plot:} 
The M plot for \(x_1\) is defined as:
$\hat{f}_{1, M}(x_1) = \mathbb{E}_{x_2 \mid x_1} \left[ \hat{f}(x_1, x_2) \mid x_1 \right].
$

Derive the mathematical expression for \(\hat{f}_{1, M}(x_1)\) and provide a step-by-step explanation of the derivation.

Hint: For two jointly normal variables \(x_1\) and \(x_2\) with means \(\mu_1, \mu_2\), variances \(\sigma_1^2, \sigma_2^2\), and covariance \(\text{Cov}(x_1, x_2) = \rho \sigma_1 \sigma_2\), the conditional expectation of \(x_2\) given \(x_1\) is: $
\mathbb{E}[x_2 \mid x_1] = \mu_2 + \rho \frac{\sigma_2}{\sigma_1} (x_1 - \mu_1).$

%This formula reflects the linear relationship between \(x_1\) and \(x_2\) in the multivariate normal case.

\item \textbf{PD Plot:} 
The Partial Dependence (PD) Plot for \(x_1\) is defined as:
$
\hat{f}_{1, \text{pdp}}(x_1) = \mathbb{E}_{x_2} \left[ \hat{f}(x_1, x_2) \right].
$
\begin{enumerate}
    \item Derive the mathematical expression for \(\hat{f}_{1, \text{pdp}}(x_1)\).
    \item Compare the PD plot with the M plot and explain any differences.
\end{enumerate}

\item \textbf{ALE Plot:}
The Accumulated Local Effect (ALE) plot for \(x_1\) is defined as:
\[
\text{ALE}_1(x_1) = \int_{z_{\min}}^{x_1} \mathbb{E}_{x_2 \mid x_1 = z} \left[ \frac{\partial \hat{f}(z, x_2)}{\partial x_1} \right] dz.
\]
The centered ALE plot is:
\[
\text{ALE}_1^{\text{centered}}(x_1) = \text{ALE}_1(x_1) - \mathbb{E}_{x_1} \left[ \text{ALE}_1(x_1) \right].
\]
\begin{enumerate}
    \item Derive the expression for the uncentered ALE plot \(\text{ALE}_1(x_1)\).
    \item Compute the centered ALE plot \(\text{ALE}_1^{\text{centered}}(x_1)\) and explain the centering process.
    \item Compare the centered ALE plot with the M and PD plots. %Discuss how they handle feature dependencies.
\end{enumerate}

\end{enumerate}
}
