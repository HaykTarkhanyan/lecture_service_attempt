\bonusaufgabe{Interactions in EBMs}{\label{ex:GAM_interactions_EBM_2}

This exercise is a continuation of exercise \ref{ex:GAM_interactions_EBM_1}.
We are concerned with the same three data sets again, but now we want to use explainable boosting machines (EBMs) to fit the data as well as possible.

\begin{enumerate}[a)]

    \item
    Similarly as in exercise \ref{ex:GAM_interactions_EBM_1}, fit an EBM model to the three different data sets.
    Evaluate its performance and compare it to the other models you have trained in exercise \ref{ex:GAM_interactions_EBM_1}.
    What conclusions would you draw regarding the general applicability of EBMs?
    
    \item
    As discussed in the lecture, the reason for EBMs being interpretable is that one can visualize all final components (univariate and interaction) of an EBM model.
    Plot these visualizations for the EBM models you have trained and compare to the visualizations of the GAMs from the previous exercise.
    What can you conclude about the data from this model?
    Did overfitting occur in any of your models?
    
\end{enumerate}

\textit{Hints:} Software packages for this exercise:

There is currently basically only one implementation of EBMs readily available inside a package, the \href{https://interpret.ml/docs/}{interpretML} python package containing EBMs \href{https://interpret.ml/docs/ebm.html}{here}.

The same package, \href{https://www.rdocumentation.org/packages/interpret/versions/0.1.34}{interpret}, also exists for R, but as of now it only contains EBMs for classification, therefore it is unfortunately not usable in this exercise.
There is, however, an R wrapper for the python package mentioned above, called \href{https://bgreenwell.github.io/ebm/reference/}{ebm}.
This package contains the main EBM functionalities from the python package.

Please note that, as this package is only a wrapper, it requires a functioning installation of the python package interpret.
If you do not have such an installation, the R wrapper package contains a function \href{https://bgreenwell.github.io/ebm/reference/install_interpret.html}{install\_interpret} for installing the python package.
Note, however, that this function automatically installs the interpret package into a separate python environment, so you need to run the R code inside this separate environment (using \href{https://packaging.python.org/en/latest/guides/installing-using-pip-and-virtual-environments/}{venv} or conda), otherwise the R wrapper will not find the python package installation.

}

