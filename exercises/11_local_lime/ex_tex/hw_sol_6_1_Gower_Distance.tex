\loesung{}{
\begin{enumerate} [a)]
  \item Derivation of Gower and feature-wise distances: 
  
  \begin{table}[ht]
\centering
\begin{tabular}{ccc|ccc|c}
  \hline
  Sex$^{(1)}$ & Sex$^{(2)}$ & $\delta_G(\textnormal{Sex}^{(1)}, \textnormal{Sex}^{(2)})$ & Age$^{(1)}$ & Age$^{(2)}$ & $\delta_G(\textnormal{Age}^{(1)}, \textnormal{Age}^{(2)})$ & $d_G(\xv^{(1)}, \xv^{(2)})$ \\
  \hline
 F & F & 0 & 15 & 15 & 0 & 0\\
 F & F & 0 & 15 & 58 & $ = \frac{1}{90-15}|15-58|= 0.573$ & 0.287\\
 F & F & 0 & 15 & 90 & 1& 0.5\\
 F & M & 1 & 15 & 15 & 0 & 0.5\\
 F & M & 1 & 15 & 58 & 0.573 & 0.787\\
 F & M & 1 & 15 & 90 & 1& 1\\
   \hline
\end{tabular}
\label{tab:probgow}
\end{table}

\item 
The table shows that the Gower distance tends to favor the observations sharing the same sex.
Two units having the same sex but largely differ in their age are considered closer than two units having the same age but a different gender. A feature-wise distance of 1 is only achieved for age if it has extreme values (90), which is rarely the case, since 
many other values could be achieved in between. While the only possible flip in sex from 
female to male, always leads to the maximum feature-wise distance of 1 - independent of 
the number of classes of the categorical features.
Outliers in age, could further aggravate this issue.

\item 
Possible weighting schemes: 
\begin{itemize}
  \item User-defined weighting schemes depending on the use case: features 
  that should not change receive high weights.
  \item Weight categorical features by their number of categories, such that
  changes in categorical features with many classes lead to lower distances. 
  \item Use the weighting scheme of exercise sheet 1/exercise 3 to find a ranking of the nominal scaled features.
  Rank them and treat them as numerical features, where $R_j$ is equal to the maximum rank.
  \item ...
\end{itemize}

\end{enumerate}
}
