\aufgabe{}{

An insurance company wants to calculate monthly premiums for a disability 
insurance ("Berufs\-unfähigkeitsversicherung") with an ML model based on the 
pension, age, job type and marital status.
An appropriate model $\fh$ was already fitted using a large customer dataset but in order 
to be launched it needs approval by regulators. 
The regulators put the model to test by using LIME (with exponential kernel) to provide an explanation
for typical and critical customers. 
One of these test instances $\xv$ is a 21 year old woman displayed in the first row of the following table.

\begin{enumerate}[a)]
  \item The regulators already generated new instances to fit the 
  surrogate model. For simplicity, 
  we assume that three instances are enough to fit the model.
  Fill out the missing fields in the following table, where $\phi_{\sigma}(\zv) = \exp(-d(\xv, \zv)^2/ \sigma^2)$ is the exponential kernel similarity measure for a specific $\sigma$ and $d(\cdot)$ as the Gower's distance.

  \begin{table}[ht]
  \centering
    \begin{tabular}{l|llll|l|l|l|l}
            & pension & age & job type       & marital status  & $\fh$ & $d(\xv, \zv_{\cdot})$ & $\phi_{\sigma = 0.15}(\zv_{\cdot})$ & $\phi_{\sigma = 0.5}(\zv_{\cdot})$ \\
            \hline
    $\xv$   & 1800    &  21 & sedentary      & single          & 30.6    & -    &  - & - \\
    $\zv_1$ & 1600    &  21 & sedentary      & married         & 25.8    & 0.25 &  0.06 &  \\
    $\zv_3$ & 2200    &  32 & sedentary      & married         & 85.2    & 0.32 &  0.01 &  \\
    $\zv_2$ & 1200    &  23 & physically     & single          & 74.9    & 0.49 &  0.00  & 0.38 \\
    \end{tabular}
  \end{table}
  
  How does the kernel width $\sigma$ influence the proximity measure? 
  What would happen if the kernel width is set too small?
  \newpage
  \item The regulatories fit two different local surrogate models ($g_1$ and $g_2$) on the re-weighted 
  data (here, three observations).
  The following table compares the prediction of $\fh$ to the ones of the two surrogate models for the three instances.
  
   \begin{table}[ht]
  \centering
    \begin{tabular}{l|l|l|l}
           & $\fh$ & $g_1$ & $g_2$    \\
           \hline
    $\xv$   & 30.6    & 34.8 &  31.1  \\
    $\zv_1$ & 25.8    & 28   &  26.1  \\
    $\zv_3$ & 85.2    & 105  &  92.7  \\
    $\zv_2$ & 74.9    & 90   &  68.9  \\
    \end{tabular}
  \end{table}
  
  Which of the two surrogate models do you prefer? 
  Compute the local faithfulness for both surrogate models using the weights from a) with $\sigma = 0.15$. \\
  \textit{Hint}: consider $L(\fh, g, \phi_{\xv})$ from the lecture.
  
  \item The surrogate model $g_1$ corresponds to a linear model using all three features, while 
  $g_2$ corresponds to a random forest with 500 trees.
  Would you still prefer the model you chose in b)? 
  
  \item Discuss whether for the faithfulness assessment in b) it makes sense to use a new sampled dataset 
  instead of the one the local surrogate model was fitted on.
  
\end{enumerate}
}
